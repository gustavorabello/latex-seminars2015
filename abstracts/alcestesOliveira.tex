
\section{Application of Finite Element Method in the study of reactive flows}

\textbf{Alcéstes Oliveira}\\
\texttt{\small{ade\_oliveira@hotmail.com}}\\
PPG-EM / UERJ

Finite Element Method (FEM) is employed to the numerical investigation of 1D and 2D reactive flows with application on determination of concentration profiles of chemical species in continuous tubular reactors which are degradable in water courses. The problem is modeled by transport equation subject to transient boundary conditions, as it is in the operation of diversified production chemical reactors and in non-uniform discharge of pollutants. Keeping the problem within certain parameters allowed for the application of the Galerkin FEM for spatial discretization and Crack-Nicolson for time discretization, overcoming stability issues and constituting a new approach for dealing with natural boundary condition, which can also contribute to increase stability of the scheme.

