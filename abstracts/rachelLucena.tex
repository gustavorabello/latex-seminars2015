
\section{Linear Stability Analysis of Fingering in Convective Dissolution in Porous Media}

\textbf{Rachel Lucena}\\
\texttt{\small{rachel.lucena@gmail.com}}\\
PPG-EM / UERJ

Fingering refers to hydrodynamic instabilities of deforming interfaces
into fingers during the displacement of fluids in porous media. The
phenomenon occurs in a variety of applications, including CO2
sequestration techniques, secondary and tertiary crude oil recovery,
fixed bed regeneration chemical processing, hydrology, filtration,
liquid chromatography, and medical applications, among others. We
consider the problem of buoyancy-driven fingering generated in porous
media by the dissolution of a fluid layer initially placed over a less
dense one in which it is partially miscible. The focus is on the lower
layer only where the convective dissolution dynamics takes place. A 2D
time dependent numerical simulation is performed, assuming that the flow
is governed by Darcy's law, along with the Boussinesq approximation to
account for buoyancy effects introduced by a concentration dependent
density. The viscosity is assumed as constant. A vorticity-stream
function formulation is adopted to solve the hydrodynamic field.

