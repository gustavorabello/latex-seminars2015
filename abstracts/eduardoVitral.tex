
\section{Nano-patterning of surfaces by ion sputtering - numerical study of the Kuramoto-Sivashinsky equation}

\textbf{Eduardo Vitral}\\
\texttt{\small{eduardo.vitral@gmail.com}}\\
PPG-EM / UERJ

Ion beam sputtering is one important technology which operates in nonequilibrium conditions and allows the processing of materials and structures outside the limits of the equilibrium thermo-dynamics. Our effort aims toward the implementation of a numerical scheme to solve a model proposed to the ion beam sputtering erosion. The phenomenon consists on the ionic bombardment of a surface, spontaneously developing a well-ordered periodicity over a large area under certain conditions. This physical process responsible for the formation of periodic structures on the previously surface is called sputtering. Depending on the energy of the incident ion, a train of collision event may be established, resulting in the ejection of atoms from the matrix. The morphology of the surface can drastically change due to these sputtered atoms, being responsible for the appearance of unexpectedly organized patterns, such as ripples and hexagonal arrays of nanoholes. In the present endeavor, a finite difference semi-implicit splitting scheme of second order in time and space is proposed to numerically solve an anisotropic Kuramoto-Sivashinsky equation subjected to periodical boundary conditions for two dimensional surfaces.

