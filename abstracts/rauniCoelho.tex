
\section{Development of a SAXS equipment for nanomaterials characterization}

\textbf{Rauni Coelho}\\
\texttt{\small{rauni.coelho@gmail.com}}\\
FEN / UERJ

With the increase in industry application of nanomaterials, the interest on equipment and techniques that can support the determination of nanoscale properties is growing. Hence, SAXS (Small Angle X-Ray Scattering) techniques allow for the analysis of nanomaterials and determine several parameters such particle size, nanoparticle density and morphology. Usually, X-Ray penetrates through the sample (transmission mode) and each particle interacts com the X-Ray emitting a signal, which detected and analyzed. As in any other research field, there are great challenges in the development of instrumentation for the application of this technique. The challenges in the present case consist of optics design, based on the platform of a conventional X-Ray diffraction equipment. The X-Ray beam must have a minimum attenuation and this condition is achieved with the evacuation of the whole optical path which includes the chamber where the sample and the gas X-Ray bi-dimensional detector are deposited.

