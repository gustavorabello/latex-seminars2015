
\section{Modeling and simulation of polydispersed multiphase flow}

\textbf{Fabio Santos}\\
\texttt{\small{fsantos@peq.coppe.ufrj.br}}\\
FEN / UERJ

Polydispersed multiphase flows are present in several natural and industrial processes, and involve a series of physical phenomena, such as: transfer of mass, momentum and energy. In bubble column chemical that are used in the biochemical and petrochemical industries, reactor efficiency significantly depends on interfacial area of the bubbles and the resident time. Therefore, the particle size distribution (PSD) is a parameter whose behavior is important to control this process. In material science, the precipitation reaction is another good example of polydispersed multiphase flow. In this case, reaction happens in a liquid phase with some chemical substances that react to form a solid with some specific features. The final market value of the crystallized product is strongly dependent on its PSD. For these reasons, modeling and simulation of  polydispersed multiphase  flow is critically important. However, the computational task is very complicated and demands special models, numerical techniques and algorithms. In this talk, I will present a computational framework to simulate polydispersed multiphase flows. I will describe models based on population balance equations (PBE) and their physical meaning. I will be also discussing one of the suitable numerical methods to couple the solution of PBE with CFD simulations. Finally, I will show some results, including parallelization of the PBE solution methods using a GPU computing paradigm.

