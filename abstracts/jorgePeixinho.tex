
\section{Recent observation in the transition to turbulence in straight, diverging and expansion pipe flow}

\textbf{Jorge Peixinho}\\
\texttt{\small{jorge.m.peixinho@gmail.com}}\\
French National Center for Scientific Research

The results of a combined experimental and numerical investigation on the transition to turbulence in a straight, diverging and expansion pipe flow of circular cross-section will be presented. First, some results for the flow in straight pipe will be recalled. Then, for diverging and sudden expansion pipe flow, the effect of the change in cross-section induces the appearance of a recirculation region. Here, at the inlet, a parabolic velocity profile is applied together with a finite amplitude perturbation to represent experimental imperfections. Initially, at low Reynolds number, the solution is steady. As the Reynolds number is increased, the length of the recirculation region near the wall grows linearly. Then, at a critical Reynolds number, a symmetry-breaking bifurcation occurs, where linear growth of asymmetry is observed. Near to the point of transition to turbulence, the flow experiences oscillations due to a shear layer instability for a narrow range of Reynolds numbers. At higher Reynolds numbers the recirculation region breaks into a turbulent state that remains spatially localised even when the perturbation is removed from the flow. The localised turbulence shows absence of metastability. Spatial correlation analysis suggests that the localised turbulence in the gradual expansion possess a different flow structure from the turbulent puff of uniform pipe flow.

