
\section{Numerical methods for data transfer during remeshing and ALE computations - application to friction stir welding process with complex geometry}

\textbf{Philippe Bussetta}\\
\texttt{\small{philippe.bussetta@gmail.com}}\\
UNESP

Friction Stir Welding (FSW) is a solid-state joining process during which materials to be joined are not melted. During the FSW process, the behaviour of the material is at the interface between solid mechanics and fluid mechanics. A 3D numerical model is presented. This model use advanced numerical techniques such as the Arbitrary Lagrangian Eulerian (ALE) formulation and remeshing operation. In both advanced numerical techniques, the method used to transfer information from one mesh (named the old mesh) to another one (called the new mesh) is an important piece of the computational process. Two data transfer methods are presented. The first method takes advantage of the properties of the ALE formalism to minimize the CPU time. The second one is a general algorithm which can be used during a complete remeshing procedure. Both data transfer methods are based on a linear reconstruction of the transferred fields over an auxiliary finite volume mesh. These data transfer procedures are applicable to both nodal values and unknowns computed at the quadrature points. These two data transfer methods are compared with the simplest transfer method, which consists of a classical interpolation.

