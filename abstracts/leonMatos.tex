
\section{Passive Cooling System}

\textbf{Leon Lima}\\
\texttt{\small{matosleon@gmail.com}}\\
PPG-EM / UERJ

Passive Cooling Systems (PCS') are engineering solutions to perform the function of heat transfer using the temperature difference between hot and cold sources to generate the driving force. Because they don't need active components to operate, PCS' have the advantages of lower costs and higher reliability. Nowadays, PCS' find large applicability in cooling functions of electronic components and in the nuclear industry. PCS' can be classified as single-phase and two-phase systems. There is a third class which operate at very high temperatures and pressures: the supercritical systems, which are single-phase with characteristics of two-phase systems. Nevertheless, independent of the type, all PCS' have the disadvantage of being subjected to instabilities, which may lead to inadmissible levels of vibrations and generate high temperature spots in the circuit. Although two-phase systems are much more susceptible to instabilities, there are conditions in which single-phase systems can be unstable.

