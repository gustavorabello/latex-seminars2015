
\section{Uncertainties in physical systems: why to quantify and how to model?}

\textbf{Americo Cunha}\\
\texttt{\small{americo@ime.uerj.br}}\\
IME / UERJ

Computational models have been increasingly used in engineering and sciences for design and analysis of complex physical systems. This increase has taken place due to the versatility and low cost of a numerical simulation compared to an approach based on experimental analyzes on a test rig. However, any computational model is subject to a series of uncertainties, due to variabilities on its parameters and, mainly, because of assumptions made in the model conception that may not be in agreement with reality. The first source of uncertainty is inherent limitations in measurement processes, manufacturing etc., while the second source is essentially due to lack of knowledge about the phenomena observed in the physical system.  An increasingly frequent requirement in engineering is the robust design of a certain component, i.e., with low sensitivity to the variation of a certain parameter, and this requires the quantification of model uncertainties. In this talk we will expose the fundamental notions related to the quantification of uncertainties in physical systems and illustrate the construction of a probabilistic model for uncertainties description in a simplistic mechanical system.

